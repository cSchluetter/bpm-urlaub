\section{Fazit}
Als Ergebnis der Realisierung ist eine prototypische Anwendung entstanden, die den Urlaubsgenehmigungsprozess in einem BPMN-Diagramm abbildet und über eine in Swing realisierte Benutzeroberfläche die Interaktion mit dem Endanwender ermöglicht.

Zu Beginn standen wir vor der Entscheidung unsere Aufgabe innerhalb der KIE Workbench, die uns auf den ersten Blick gut gefallen hat, zu realisieren. Durch die Anmeldung unterschiedlicher Benutzer konnte der Ablauf von Prozessen ziemlich realitätsnah simuliert werden. Da der Prozessdesigner jedoch mehrfach fehlerhaftes XML erzeugt hat und damit das BPMN-Diagramm nicht mehr anzeigen konnte haben wir unsere Anwendung letztendlich in Eclipse und dem BPMN Modeler 2.0 Plugin realisiert.

Das Eclipse-Plugin hat uns eine einfache Möglichkeit gegeben den Prozess in einem grafischen Prozessdesigner zu gestalten. Besonders gut gefallen hat uns die Tatsache, dass wir die entsprechenden WorkItemHandler in Java implementieren konnten. So konnten wir alle externen Abhängigkeiten hinter Schnittstellen verstecken und deren Implementierung einfach austauschen. Beispielsweise haben wir zu Beginn unseres Projektes alle Eingaben über die Konsole gemacht und erst später eine Benutzeroberfläche in Swing realisiert. Ohne dass wir den bestehenden Code anpassen mussten konnten wir hier die Implementierung wechseln.

Bei der Arbeit mit asynchronen Callback Funktionen ist uns aufgefallen, dass der Logger nicht mehr protokolliert hat. Dieser scheint mit Asynchronität so seine Probleme zu haben.

Insgesamt haben wir für uns festgestellt dass sich Geschäftsprozesse relativ gut abbilden lassen. Auch die Implementierung eigener "`Custom Tasks"' stellt kein großes Problem da. So ist die Entwicklungsumgebung flexibel und bietet alle nötigen Möglichkeiten einen vollständigen Geschäftsprozess digital abzubilden.

Über den aktuellen Stand der derzeitigen Implementierung hinaus könnte man sich noch eine Persitierung der Daten vorstellen. Alternativ könnte man gewöhnliche Services aus der echten Welt mit einbinden und beispielsweise die Vorgesetzten über das Active Directory ermitteln. Die Benachrichtigung könnte man in Echt über E-Mails realisieren.

Unsere prototypische Anwendung erfüllt alle Anforderungen und lässt sich komfortabel bedienen.